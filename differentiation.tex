% to be included
\section{Differentiation}

Here we will study differentiation on smooth manifolds. The smooth structure locally comes from the sheaf of smooth functions on $\R^n$, which we denote by $C^\infty(\R^n)$.

\begin{definition}
    A \emph{differential manifold $M$ of dimension $n$} is a Hausdorff and second-countable topological space, together with a $\R$-algebra subsheaf of the sheaf of continuous functions, called the \emph{structure sheaf} $\mathcal{A}^M$ of $M$, such that $\forall x\in M$ there exists an open neighborhood $U$ of $x$ with a homeomorphism $\varphi$ into an open subset of $\R^n$, satisfying $\mathcal{A}^M|U=\varphi^{-1}C^\infty(\R^n)$. Such $(U,\varphi)$ is called a \emph{coordinate chart}.
\end{definition}

\begin{proposition}\label{prop.maniparacompact}
    Any manifold $M$ is paracompact and normal.
\end{proposition}

\begin{proof}[Proof(sketch)]
    A manifold is locally compact and Hausdorff, hence regular. The Urysohn metrization theorem states that every second-countable regular Hausdorff space is metrizable. It is clear that metric spaces are normal. Furthermore, they are paracompact; see \href{https://www.ams.org/journals/proc/1969-020-02/S0002-9939-1969-0236876-3/S0002-9939-1969-0236876-3.pdf}{here} for a neat proof by Rudin which provides a refinement of a well-ordered cover.
\end{proof}

\begin{definition}
    By definition, any open subset of $\R^n$ is a differentiable manifold of dimension $n$. Slightly more generally, any open subset of a differentiable manifold $(M,\mathcal{A})$ is again a differentiable manifold of the same dimension, and is called an \emph{open submanifold} of $M$.
\end{definition}

Suppose $X$ is a Hausdorff second-countable space and $\{U_i\}$ is an open cover. By Proposition \ref{prop.sheafglue}, a given collection of differentiable structures $(U_i,\mathcal{A}_i)$ can be glued together if $\mathcal{A}_i|U_i\cap U_j=\mathcal{A}_j|U_i\cap U_j$ as a subsheaf of the sheaf of continuous functions on $U_i\cap U_j$. In other words, $U_i\cap U_j$ is the same manifold as open submanifold of $U_i$ or $U_j$.

Indeed $\mathcal{A}_i$ can be given by a homeomorphism $\varphi_i$ from $U_i$ to an open subset of $\R^n$. The condition becomes checking that  $\varphi_i\circ\varphi_j^{-1}$ are all diffeomorphisms. Then ($U_i$,$\varphi_i$) become the coodinate charts. This is of course the alternate definition of differential manifolds.

\begin{example}\label{eg.projspace}
    Let $F=\R$ or $\C$, and $V$ be a $F$-vector space with $n=\dim_FV$. We endow the projective space $P(V):=(V\setminus\{0\})/F^\times$ with a differentiable structure, induced by
    \begin{align*}
        \{V:x_i\neq0\} & \longrightarrow F^{n-1} \\
        (x_1,\cdots,x_i,\cdots) & \mapsto \left(\frac{x_j}{x_i}\right)_{j\neq i}
    \end{align*}

    The compatibility condition is just the differentiability of 
    $$(y_j)\mapsto\left(\frac{y_1}{y_i},\cdots,\frac{y_{i-1}}{y_i},\frac{1}{y_i},\frac{y_{i+1}}{y_i},\cdots\right),$$
    which is obvious. This provides differentiable structures for
    $$\mathbb{RP}^n:=P(\R^{n+1}),\text{ and }\mathbb{CP}^n:=P(\C^{n+1}).$$
\end{example}

Submanifolds need not be open. For example, $S^n\subset\R^{n+1}$ is closed. More generally, we have

\begin{definition}
    A subset $N$ of a differential manifold $M$ is called a \emph{closed submanifold}, if
    \begin{itemize}
        \item $N$ is closed, and
        \item $\forall x\in N$, there is a coordinate chart $(U,\varphi)$ around $x$ such that $U\cap N$ is the common zero of some (constant number of) coordinate functions in $\varphi$.
    \end{itemize}

    In other words, a closed submanifold is locally the inverse image of some linear subspace of $\R^n$. Apparently these charts are mutually compatible, so $N$ gets a differentiable structure from $M$.
\end{definition}

\begin{example}
    Sections of the sheaf $\mathcal{A}$ are called \emph{differentiable functions}. Let $f$ be a differentiable function on $M$. 
    
    If at every point, at least one of the partial derivatives (with respect to any coordinate chart) does not vanish, then the \emph{zero locus} $f^{-1}(0)$ has a differentiable structure as a closed submanifold, for the inverse function theorem \ref{thm.invfunc} makes $f$ a coordinate function. To be specific, if $\dfrac{\partial f}{\partial x_1}\neq 0$, then $(U,\varphi=(f,x_2,\cdots,x_n))$ is a coordinate chart for some $U$.

    If $f=(f_1,f_2,\cdots,f_r)$, which can be seen as a differentiable function into $\R^r$ (the general definition will be stated later), then again we have the analogous condition
    $$\operatorname{rank}(\nabla f)=r$$
    for $f^{-1}(0)$ to be a closed submanifold. We also see that 
    $$\dim f^{-1}(0)=\dim M-r,$$
    by the definition of closed submanifolds.
\end{example}

Take $M=\R^{n+1}$ and $f(x)=|x|^2-1$, and we are back at the example $S^n$.

If the function has critical points, the zero locus may not have a manifold structure. Consider $f:\R^2\to\R$ given by $(x,y)\mapsto xy$. Its zero locus is the union of $x,y$-axes which is not locally Euclidean.

\begin{example}
    Clearly, $\operatorname{GL}(n,\R)\subset M_n(\R)\simeq\R^{n^2}$ given by $\det\neq 0$ is an open submanifold, which has $\operatorname{SL}(n,\R)$ given by $\det=1$ as a closed submanifold of $M_n(\R)$. Here
    $$\dim\operatorname{GL}(n,\R)=n^2,\quad\dim\operatorname{SL}(n,\R)=n^2-1.$$
    The condition can be checked directly.
\end{example}

There is an interesting generalization of the projective space $P(V)$ which also has a geometric structure, called the \emph{Grassmannian}. Following the notation of Example \ref{eg.projspace}, the Grassmannian $\mathbf{G}(k,V)$, $0\leq k\leq n$, is the set of $k$-dimensional linear subspaces of $V$.

\begin{proposition}
    There is a differentiable structure on $\mathbf{G}(k,V)$, and $\dim\mathbf{G}(k,V)=k(n-k)$.
\end{proposition}

\begin{proof}
    Let $U\in\mathbf{G}(k,V)$ and $W$ be such that $V=U\oplus W$. Consider the subset $\{U':V=U'\oplus W\}$. There is the bijection
    \[\begin{tikzcd}[row sep=tiny]
        \{U':V=U'\oplus W\} \arrow[r, leftrightarrow, "1:1"] & \operatorname{Hom}(U,W) \\
        U' \arrow[r, mapsto] & \left [U\hookrightarrow V\xrightarrow{\pi}W\right ] \\
        \operatorname{im}(\iota-\varphi) \arrow[r, mapsfrom] & \varphi
    \end{tikzcd}\]
    and of course $\operatorname{Hom}(U,W)\simeq\R^{k(n-k)}$. Here $\pi$ denotes the projection with respect to $U'\oplus W$.

    Now we need to check three things: that this topology on $\mathbf{G}(k,V)$ is Hausdorff and second-countable, and that the transition maps are differentiable.

    %TODO
\end{proof}

\begin{theorem}[Plücker embedding]
    The following map is injective and called the \emph{Plücker embedding}:
    \begin{align*}
        \psi:\mathbf{G}(k,V) &\longrightarrow P(\bigwedge^kV) \\
        W &\longmapsto F^\times w_1\wedge\cdots\wedge w_k=(\bigwedge^kW)\setminus\{0\}
    \end{align*}
    where $w_1,w_2,\cdots,w_k$ is any basis of $W$.
\end{theorem}

\begin{proof}
    $w_1\wedge\cdots\wedge w_k$ is nonzero and generates $\bigwedge^kW$, so $\psi(W)\in P(\bigwedge^kV)$. To prove injectivity of $\psi$, notice that if $\psi(W)=F^\times\alpha$, then
    $$W=\{v\in V:v\wedge\alpha=0\},$$
    which is a simple consequence of the explicit basis of $\bigwedge V$ (indexed by the multi-indices $\lambda=(1\leq\lambda_1<\cdots<\lambda_k\leq n)$).
\end{proof}

It can be shown that under this map, $\mathbf{G}(k,V)$ is a closed submanifold of $P(\bigwedge^kV)$. However we leave the proof of this for later at the appropriate juncture.

We have a few more remarks on the algebra regarding the Grassmannian before we move on. %TODO contraction, duality, prepare for compute of differential forms

Closed submanifolds are the nicest 

We now shift our focus to basic properties of the differentiable functions. By the locality of sections, the following is obvious.

\begin{proposition}
    If $f\in\mathcal{A}(M)$ is nowhere vanishing, then $1/f$ is differentiable and also nowhere vanishing.

    If $\{U_i\}_i$ is a locally finite open cover of $M$, if $\supp f_i\in U_i$ for all $i$, then $\sum f_i\in\mathcal{A}(M)$.
\end{proposition}

\begin{proof}
    Trivial.
\end{proof}

\begin{definition}
    A map $\varphi:M\to N$ between differentiable manifolds is called \emph{differentiable}, if it is continuous, and if for open set $U$ in $N$ and any function $f\in\mathcal{A}_N(U)$, $f\circ\varphi\in\mathcal{A}_M(\varphi^{-1}(U))$.

    The key feature of this definition is it induces a \emph{structure homomorphism} of sheaves $\mathcal{A}_N\to \varphi_*\mathcal{A}_M$, or equivalently (Theorem \ref{thm.shf.pulbck}), a homomorphism $\varphi^{-1}\mathcal{A}_N\to\mathcal{A}_M$.

    Denote the category of differential manifolds and differentiable maps $\mathsf{Man}$. This is valid since a composition of differentiable maps is obviously differentiable. 
\end{definition}

\begin{theorem}
    $\mathsf{Man}$ possesses finite products.
\end{theorem}

\begin{proof}
    If $M$ (and $N$) are manifolds covered by coordinate charts $\{U_i\}$ (and $\{V_j\}$), the space $M\times N$ can be covered by $\{U_i\times V_j\}$. These subsets are identified with open subsets of $\R^{m+n}=\R^m\times\R^n$, and the compatibility follows from that of $\{U\}$ and $\{V\}$.

    If $A\to M$ and $A\to N$ are differentiable maps, the differentiability of the induced continuous map $A\to M\times N$ is a direct consequence of the locality of this property, namely, that it need only be checked in an open neighborhood around each point.
\end{proof}

\begin{example}
    Isomorphisms of $\mathsf{Man}$ are called \emph{diffeomorphisms}. A differentiable homeomorphism need not be a diffeomorphism: For example, $x\mapsto x^3$ is a homeomorphism $\R\to\R$ whose inverse is not differentiable at $0$.
    % Man!
\end{example}

\begin{theorem}[Differentiable partition of unity]
    If open sets $U,V$ in a manifold $M$ satisfy $\overline{V}\subset U$, then there exists a nonnegative differentiable function $f$ with support in $U$ which does not vanish anywhere in $V$.
\end{theorem}

\begin{proof}
    We can add local functions up to achieve this goal. So we take a locally finite cover of $M$ by coordinate charts $\{W_i\}$, and for each open set in $\R^n$ associated to $W_i\cap V$ take a locally finite cover by open balls. But in the ball $B:=B(a,r)\subset\R^n$, the function
    $$f(x)=\begin{cases}
        \exp\left(\frac{1}{\|x-a\|^2-r^2}\right), & x\in B \\
        0, &\text{otherwise}
    \end{cases}$$
    is non-vanishing in $B$ and $0$ everywhere else, and $\supp f\subset\overline{V}\subset U$. This completes the argument.
\end{proof}

\begin{corollary}
    Given a locally finite open cover $\{U_i\}$ of $M$, there exists functions $\{\varphi_i\}$ with values in $[0,1]$ such that $\supp\varphi_i\subset U_i$ for all $i$ and $\sum\varphi_i=1$.

    The family of functions $\{\varphi_i\}$ is called the \emph{partition of unity} subordinate to the open cover $\{U_i\}$.
\end{corollary}

\begin{proof}
    By Proposition \ref{prop.maniparacompact}, $M$ has the shrinking property. That is, we have a refinement $\{V_i\}$ such that $\overline{V_i}\subset U_i$ for all $i$. Hence there exists functions $\{f_i\}$ such that each $f_i$ has support in $U_i$ and does not vanish anywhere in $V_i$. Since $f:=\sum f_i$ is non-vanishing, $\{f_i/f\}_i$ is a partition of unity subordinate to $\{U\}$.
\end{proof}

This property is extremely useful and builds our intuition towards geometry. This is partly exhibited in the following lemma.

\begin{lemma}\label{lem.PoU.extension}
    Any section of $\mathcal{A}$ over a closed subset $K$ can be extended to a global differentiable function on $M$.

    By Theorem \ref{thm.closedsect}, this can be equivalently stated as follows: Given a differentiable function $f$ in some open neighborhood $U$ of $K$, there is a global function $\tilde f$ which coincides with $f$ in an open neighborhood of $K$.
\end{lemma}

\begin{proof}
    By Proposition \ref{prop.maniparacompact}, there is another open neighborhood $V$ of $K$ satisfying $\overline{V}\subset U$. Using partition of unity on the cover $\{U,M\setminus\overline{V}\}$, we obtain a function $\varphi$ which is $1$ on $V$ and whose support is in $U$. Then the function
    $$\tilde f=\begin{cases}
        f\varphi, & \text{on }U \\
        0, & \text{on }M\setminus(\supp\varphi)
    \end{cases}$$
    is well-defined and differentiable. It is defined on $M$, as required.
\end{proof}

\begin{proposition}
    Let $\varphi:M\to N$ be a continuous map between manifolds. If for every $f\in\mathcal{A}(N)$, the pull-back $f\circ\varphi\in\mathcal{A}(M)$, then $\varphi$ is differentiable.
\end{proposition}

\begin{proof}
    Given $U$ open in $N$ and $g\in\mathcal{A}_N(U)$, we apply Lemma \ref{lem.PoU.extension} to $\{x,g\}$, where $x\in U$ is any point, to get a corresponding global function $\tilde g$. The pull-back $\tilde g\circ\varphi$ coincides with $g\circ\varphi$ in a neighborhood of $\varphi^{-1}(x)$.
\end{proof}

