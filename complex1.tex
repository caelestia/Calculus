% to be included
\section{Complex Analysis}

We start with declaring a notion of line integral. Recall that a function $\gamma:I\to\C$ is \emph{absolutely continuous} if and only if $\gamma'$ exists a.e., $\gamma'\in L^1(I,dt)$ and
$$\gamma(s)-\gamma(0)=\int_0^s\gamma'(t)dt$$
for all $s\in I$.

\begin{definition}
    Given an absolutely continuous \emph{path} $\gamma$ and an integrable function $f:I\to\C$, we can define the line integral
    $$\int_\gamma fdz:=\int_If(t)\gamma'(t)dt.$$
    
    Sometimes we also work on the free abelian group generated by $\gamma$'s. For example, one may write
    $$\int_{\gamma-\gamma'}fdz=\int_\gamma fdz-\int_{\gamma'}fdz.$$
\end{definition}

In the majority of cases, the path is piecewise $C^1$, and the above definition is better than enough. Still, we proceed to give a more general setup. Some theorems will be proven in this fashion to demonstrate the technique and to support our intuition toward paths.

\begin{definition}[Riemann-Stieltjes integral]
    We say a curve $\gamma:I\to\C$ is \emph{rectifiable} or \emph{of bounded variation}, if
    $$\ell(\gamma):=\sup_\lambda\sum|\gamma(\lambda_i)-\gamma(\lambda_{i+1})|<\infty,$$
    where $\lambda$ runs through all partitions $(0=\lambda_0<\lambda_1<
    \cdots<\lambda_n=1)$ of $I$. $\ell(\gamma)$ is called the \emph{length} of $\gamma$.

    It is clear that $\ell(\gamma)<\infty$ if and only if both $\re\gamma$ and $\im\gamma$ have bounded variation.
\end{definition}

\begin{proposition}
    If $\gamma:I\to\C$ is piecewise $C^1$, then $\gamma$ is rectifiable and
    $$\ell(\gamma)=\int_I|\gamma'|dt$$
\end{proposition}

%

\begin{definition}
    We usually consider functions from a (open) subset $\Omega\subset\C$ to $\C$. Indeed, such a function $f$ is said to be \emph{differentiable} at $z_0\in\Omega$, if
    $$f'(z_0)=\lim_{z\to z_0}\frac{f(z)-f(z_0)}{z-z_0}$$
    exists; and \emph{holomorphic}, if it is differentiable in an open set.
\end{definition}

\begin{theorem}[C-R equation]
    If $f(x+\I y)=a(x,y)+\I b(x,y)$ is holomorphic at $z$, then 
\end{theorem}

\begin{proposition}
    If $f(x+\I y)=a(x,y)+\I b(x,y)$ is holomorphic on an open set $\Omega$, then $\triangle a=\triangle b=0$.
\end{proposition}

\begin{proposition}
    If $a:\R^2\to\R$ satisfies $\triangle a=0$ on an open set $\Omega$,then there exists holomorphic $f:\Omega\to\C$ s.t. $a=\operatorname{Re}f$.
\end{proposition}