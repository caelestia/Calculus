% to be included
\section[Section Title. Section Subtitle]{Complex Analysis\\ {\large Cauchy's Theorem}}

In this section, we discuss the basics of holomorphic functions, which behave extraordinarily well as we know.

\begin{definition}
    We usually consider functions from a (open) subset $\Omega\subset\C$ to $\C$. Indeed, such a function $f$ is said to be \emph{differentiable} at $z_0\in\Omega$, if
    $$f'(z_0)=\lim_{z\to z_0}\frac{f(z)-f(z_0)}{z-z_0}$$
    exists; and \emph{holomorphic}, if it is differentiable in an open set.
\end{definition}

\begin{theorem}[C-R equation]
    If $f(x+\I y)=a(x,y)+\I b(x,y)$ is holomorphic at $z$, then 
\end{theorem}

\begin{proposition}
    If $f(x+\I y)=a(x,y)+\I b(x,y)$ is holomorphic on an open set $\Omega$, then $\triangle a=\triangle b=0$.
\end{proposition}

\begin{proposition}
    If $a:\R^2\to\R$ satisfies $\triangle a=0$ on an open set $\Omega$,then there exists holomorphic $f:\Omega\to\C$ s.t. $a=\operatorname{Re}f$.
\end{proposition}